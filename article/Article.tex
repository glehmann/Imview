%
% Complete documentation on the extended LaTeX markup used for Insight
% documentation is available in ``Documenting Insight'', which is part
% of the standard documentation for Insight.  It may be found online
% at:
%
%     http://www.itk.org/

\documentclass{InsightArticle}


%%%%%%%%%%%%%%%%%%%%%%%%%%%%%%%%%%%%%%%%%%%%%%%%%%%%%%%%%%%%%%%%%%
%
%  hyperref should be the last package to be loaded.
%
%%%%%%%%%%%%%%%%%%%%%%%%%%%%%%%%%%%%%%%%%%%%%%%%%%%%%%%%%%%%%%%%%%
\usepackage[dvips,
bookmarks,
bookmarksopen,
backref,
colorlinks,linkcolor={blue},citecolor={blue},urlcolor={blue},
]{hyperref}
% to be able to use options in graphics
\usepackage{graphicx}
% for pseudo code
\usepackage{listings}
% subfigures
\usepackage{subfigure}


%  This is a template for Papers to the Insight Journal. 
%  It is comparable to a technical report format.

% The title should be descriptive enough for people to be able to find
% the relevant document. 
\title{}

% Increment the release number whenever significant changes are made.
% The author and/or editor can define 'significant' however they like.
\release{0.00}

% At minimum, give your name and an email address.  You can include a
% snail-mail address if you like.
\author{Richard Beare}
\authoraddress{Richard.Beare@ieee.org}

\begin{document}
\maketitle

\ifhtml
\chapter*{Front Matter\label{front}}
\fi


\begin{abstract}
\noindent
% The abstract should be a paragraph or two long, and describe the
% scope of the document.
{\tt Imview}\cite{imview} is a handy image viewer with functionality oriented
towards image analysis tasks. It can display a wide variety of image
and pixel types, including 3D image, and includes functions for
measurement, overlay display and limited interaction via pointfile. It
also has an image server capability so that image data can be
transmitted via a shared memory or socket interface so that images can
be displayed without worrying about file type restrictions.

This article describes the beginning of an interface between {\bf
WrapITK} and imview.
\end{abstract}

\tableofcontents

\section{Installation}
This package can be built as an external project for WrapITK
\cite{WrapITK}. WrapITK must be installed first, then follow 
the instructions for installing an external project.

\section{Examples of usage}
This set of commands is also available the file {\tt testing.py} in the root of the repository

Import the necessary modules:
\begin{verbatim}
import itk
import Imview
\end{verbatim}


Declare some image types:
\begin{verbatim}
t1=itk.Image[itk.UC, 2]
t2=itk.Image[itk.US, 2]
t3=itk.Image[itk.UC, 3]
t4=itk.Image[itk.F, 3]
\end{verbatim}

Load some images
\begin{verbatim}
reader1=itk.ImageFileReader[t1].New(FileName='images/cthead1.png')
reader2=itk.ImageFileReader[t2].New(FileName='images/cthead1.png')
reader3=itk.ImageFileReader[t3].New(FileName='images/ESCells.img')
reader4=itk.ImageFileReader[t4].New(FileName='images/ESCells.img')
\end{verbatim}

Perform some arithmetic operations so the dynamic range changes.
\begin{verbatim}
m1 = itk.MultiplyImageFilter[t2, t2, t2].New(reader2, reader2)
f1 = itk.CosImageFilter[t4,t4].New(reader4)
\end{verbatim}

Perform thresholding - results will be used as overlay masks.
\begin{verbatim}
# a 2d threshold image
thresh1 = itk.OtsuThresholdImageFilter[t1, t1].New(reader1)
thresh1.SetInsideValue(1)
thresh1.SetOutsideValue(0)

# a 3d threshold image
thresh2 = itk.OtsuThresholdImageFilter[t3, t3].New(reader3)
thresh2.SetInsideValue(1)
thresh2.SetOutsideValue(0)
\end{verbatim}

Create an imview instance and display some images. Pressing the space
bar will toggle between the images.
\begin{verbatim}
v1=itk.imview()
v1.Show(reader1, title="char image")
v1.Show(m1, title="short image")
\end{verbatim}

Display 3D images. Space bar will toggle between the two and {\tt
insert} and {\tt delete} will move between planes.
\begin{verbatim}
v2 = itk.imview(reader3, title="ESCells raw")
v2.Show(f1, title="3d float")
\end{verbatim}

Display overlays on the image. Overlays will appear on the currently
visible image. Dimensions must match.
\begin{verbatim}
v1.Overlay(thresh1)
v2.Overlay(thresh2)
\end{verbatim}

Create another instance.
\begin{verbatim}
v1a=itk.imview(reader1)
\end{verbatim}

Change the colourmap. Options can also be selected from imview menus.
\begin{verbatim}
v1a.ColorMap("inv_spiral.lut")
\end{verbatim}

Link this instance with another. After linking zooming and panning
operations will be synchronised. Check the imview docs about how to
zoom with the mouse.
\begin{verbatim}
v1a.Link([v1])
\end{verbatim}
Close the overlay.
\begin{verbatim}
v1.CloseOverlay()
\end{verbatim}

Retrieve marked points - Shift - click combinations mark points.
\begin{verbatim}
pts = v1.GetPointfile
\end{verbatim}


\appendix



\bibliographystyle{plain}
\bibliography{Article,InsightJournal}
\nocite{ITKSoftwareGuide}

\end{document}

